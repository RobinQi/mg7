%%%%%%%%%%%%%%%%%%%%%%%%%%%%%%%%%%%%%%%%%%%%%%%%%%%%%%%%%%%%%%%%%%%%%%%%%%%%%%%%%%%%%%%%%%%%%%%%%%%%%%%%%%%%%%%%%%%%%%%%%%%%%%%%%%%%%%%%%%%%%%%%%%%%%%%%%%%
% This is just an example/guide for you to refer to when submitting manuscripts to Frontiers, it is not mandatory to use Frontiers .cls files nor frontiers.tex  %
% This will only generate the Manuscript, the final article will be typeset by Frontiers after acceptance.                                                 %
%                                                                                                                                                         %
% When submitting your files, remember to upload this *tex file, the pdf generated with it, the *bib file (if bibliography is not within the *tex) and all the figures.
%%%%%%%%%%%%%%%%%%%%%%%%%%%%%%%%%%%%%%%%%%%%%%%%%%%%%%%%%%%%%%%%%%%%%%%%%%%%%%%%%%%%%%%%%%%%%%%%%%%%%%%%%%%%%%%%%%%%%%%%%%%%%%%%%%%%%%%%%%%%%%%%%%%%%%%%%%%

%%% Version 3.1 Generated 2015/22/05 %%%
%%% You will need to have the following packages installed: datetime, fmtcount, etoolbox, fcprefix, which are normally inlcuded in WinEdt. %%%
%%% In http://www.ctan.org/ you can find the packages and how to install them, if necessary. %%%

\documentclass{frontiersSCNS} % for Science, Engineering and Humanities and Social Sciences articles
%\documentclass{frontiersHLTH} % for Health articles
%\documentclass{frontiersFPHY} % for Physics and Applied Mathematics and Statistics articles

%\setcitestyle{square}
\usepackage{url,hyperref,lineno,microtype}
\usepackage[onehalfspacing]{setspace}
\linenumbers

% fix for pandoc 1.14
\providecommand{\tightlist}{%
\setlength{\itemsep}{0pt}\setlength{\parskip}{0pt}}

% Leave a blank line between paragraphs instead of using \\


\def\keyFont{\fontsize{8}{11}\helveticabold }
\def\firstAuthorLast{Alexey Alekhin {et~al.}} %use et al only if is more than 1 author
\def\Authors{
 Alexey Alekhin\(^\dagger\)\,$^{1}$  Evdokim Kovach\(^\dagger\)\,$^{1}$  Marina Manrique\,$^{1}$  Pablo Pareja\,$^{1}$  Eduardo Pareja\,$^{1}$  Raquel Tobes\,$^{1}$  and Eduardo Pareja-Tobes $^{1,*}$
}% Affiliations should be keyed to the author's name with superscript numbers and be listed as follows: Laboratory, Institute, Department, Organization, City, State abbreviation (USA, Canada, Australia), and Country (without detailed address information such as city zip codes or street names).
% If one of the authors has a change of address, list the new address below the correspondence details using a superscript symbol and use the same symbol to indicate the author in the author list.
\def\Address{$^{1}$Oh no sequences! Research Group, Era7 Bioinformatics, Granada, Spain}
% The Corresponding Author should be marked with an asterisk
% Provide the exact contact address (this time including street name and city zip code) and email of the corresponding author
\def\corrAuthor{Corresponding Author}
\def\corrAddress{Oh no sequences! Research Group, Era7 Bioinformatics, Plaza Campo Verde 3, Granada, 18001, Spain}
\def\corrEmail{eparejatobes@ohnosequences.com}


\begin{document}
\onecolumn
\firstpage{1}

\title[MG7: Configurable and scalable 16S metagenomics data analysis]{MG7: Configurable and scalable 16S metagenomics data analysis -- new
methods optimized for massive cloud computing}

\author[\firstAuthorLast ]{\Authors} %This field will be automatically populated
\address{} %This field will be automatically populated
\correspondance{} %This field will be automatically populated

\extraAuth{}% If there are more than 1 corresponding author, comment this line and uncomment the next one.
%\extraAuth{corresponding Author2 \\ Laboratory X2, Institute X2, Department X2, Organization X2, Street X2, City X2 , State XX2 (only USA, Canada and Australia), Zip Code2, X2 Country X2, email2@uni2.edu}

\maketitle

\begin{abstract}

No abstract yet. Will be here.

\(^\dagger\) The first and second authors contributed equally to this
work

\tiny
 \keyFont{ \section{Keywords:} Metagenomics, 16S, Bacterial diversity profile, Bio4j, Graph databases,
Cloud computing, NGS, Genomic big data } %All article types: you may provide up to 8 keywords; at least 5 are mandatory.
\end{abstract}

\section{Introduction}\label{introduction}

During the past decade, metagenomics data analysis is growing
exponentially. Some of the reasons behind this are the increasing
throughput of massively parallel sequencing technologies (with the
derived decrease in sequencing costs), and the wide impact of
metagenomics studies, especially in human health (diagnostics,
treatments, drug response or prevention). We should also mention what
could be called the microbiome explosion: all kind of microbiomes (gut,
mouth, skin, urinary tract, airway, milk, bladder) are now routinely
sequenced in different conditions of health and disease, or after
different treatments. The impact of Metagenomics is also being felt in
environmental sciences, crop sciences, the agrifood sector and
biotechnology in general. These new possibilities for exploring the
diversity of micro--organisms in the most varied environments are
opening new research areas, and drastically changing the existing ones.

As a consequence, the challenge is thus moving (as in other fields) from
data acquisition to data analysis: the amount of data is expected to be
overwhelming in a very short time \citep{stephens2015big}.

Genome researchers have raised the alarm over big data in the past
\citep{hayden2015genome}, but even a more serious challenge might be
faced with the metagenomics boom. If we compare metagenomics data with
other genomics data used in clinical genotyping we find a differential
feature: the key role of time. Thus, for example, in some longitudinal
studies, serial sampling of the same patient along several weeks (or
years) is being used for the follow up of some intestinal pathologies,
for studying the evolution of the gut microbiome after antibiotic
treatment, or for colon cancer early detection
\citep{zeller2014potential}. This need of sampling across time adds more
complexity to metagenomics data storage and demands adapted algorithms
to detect state variations across time as well as idiosyncratic
commonalities of the microbiome of each individual
\citep{franzosa2015identifying}. In addition to the intra-individual
sampling-time dependence, metagenomic clinical test results vary
depending on the specific region of extraction of the clinical specimen.
This local variability adds complexity to the analysis since different
localizations (different tissues, different anatomical regions, healthy
or tumour tissues) are required to have a sufficiently complete
landscape of the human microbiome. Moreover, reanalysis of old samples
using new tools and better reference databases might be also demanded
from time to time.

During the last years other sciences as astronomy or particle physics
are facing the big data challenge but, at least, these science have
standards for data processing \citep{stephens2015big}. Global standards
for converting raw sequence data into processed data are not yet well
defined in metagenomics and there are shortcomings derived from the fact
that many bioinformatics methodologies currently used for metagenomics
data analysis were designed for a scenario very different that the
current one. These are some of the aspects that have suffered crucial
changes and advances with a direct impact in metagenomics data analysis:

\begin{enumerate}
\def\labelenumi{\roman{enumi}.}
\tightlist
\item
  The first aspect is related to the sequences to be analyzed: the reads
  are larger, the sequencing depth and the number of samples of each
  project are considerably bigger. The first metagenomics studies were
  very local projects, while nowadays the most fruitful studies are done
  at a global level (international, continental, national). This kind of
  global studies has yielded the discovery of clinical biomarkers for
  diseases of the importance of cancer, obesity or inflammatory bowel
  diseases and has allowed exploring the biodiversity in many earth
  environments
\item
  The second aspect derives from the impressive genomics explosion, its
  effect being felt in this case in the reference sequences. The immense
  amount of sequences available in public repositories demands new
  approaches in curation, update and storage for metagenomics reference
  databases: current models will or already have problems to face the
  future avalanche of metagenomic sequences.
\item
  The third aspect to consider for metagenomics data analysis is related
  to the appearance of new models for massive computation and storage
  such as the so-called cloud, or the widespread adoption of programming
  methodologies like functional programming, or, more speculatively,
  dependently typed programming. The immense new possibilities that
  these advances offer must have a direct impact in metagenomics data
  analysis.
\item
  And finally the new social manner to do science, and especially
  genomic science is the fourth aspect to consider. Metagenomics evolves
  in a social and global scenario following a science democratization
  trend in which many small research groups from distant countries share
  a common big metagenomics project. This global cooperation demands
  systems allowing following exactly the same pipelines using equivalent
  cloud resources to modularly execute the analysis in an asynchronous
  way of working between different groups. This new scenario calls for
  new methods and tools to handle the current and future volume of
  metagenomic data with the sufficient speed of analysis.
\end{enumerate}

\section{Results}\label{results}

\subsection{Overview}\label{overview}

Considering the current new metagenomics scenario and to tackle the
challenges posed by metagenomics big data analysis outlined in the
Introduction we have designed a new open source methodology for
analyzing metagenomics data. It exploits the new possibilities that
cloud computing offers to get a system robust, programmatically
configurable, modular, distributed, flexible, scalable and traceable in
which the biological databases of reference sequences can be easily
updated and/or frequently substituted by new ones or by databases
specifically designed for focused projects.

These are some of the more innovative MG7 features:

\begin{itemize}
\tightlist
\item
  Static reproducible specification of dependencies and behavior of the
  different components using \emph{Statika} and \emph{Datasets}
\item
  Parallelization and distributed analysis based on AWS, with on-demand
  infrastructure as the basic paradigm
\item
  Definition of complex workflows using \emph{Loquat}, a composable
  system for scaling/parallelizing stateless computations especially
  designed for Amazon Web Services (AWS)
\item
  A new approach to data analysis specification, management and
  specification based on working with it in exactly the same way as for
  a software project, together with the extensive use of compile-time
  structures and checks
\item
  Modeling of the taxonomy tree using the new paradigm of graph
  databases (Bio4j). It facilitates the taxonomic assignment tasks and
  the calculation of the taxa abundance values considering the
  hierarchic structure of taxonomy tree (cumulative values)
\item
  Exhaustive per-read taxonomic assignment using two complementary
  assignment algorithms Lowest Common Ancestor and Best BLAST Hit
\item
  Using a new 16S database of reference sequences (16S-DB7) with a
  flexible and sustainable system of updating and project-driven
  customization
\end{itemize}

\subsection{Libraries and resources}\label{libraries-and-resources}

In this section we describe the resources and libraries developed by the
authors on top of which MG7 is built. All MG7 code is written in
\href{http://www.scala-lang.org/}{Scala}, a hybrid object-functional
programming language. Scala was chosen based on the possibility of using
certain advanced programming styles, and Java interoperability, which
let us build on the vast number of existing Java libraries; we take
advantage of this when using Bio4j as an API for the NCBI taxonomy. It
has support for type-level programming, type-dependent types (through
type members) and singleton types, which permits a restricted form of
dependent types where types can depend essentially on values determined
at compile time (through their corresponding singleton types).
Conversely, through implicits one can retrieve the value corresponding
to a singleton type.

\subsubsection{\texorpdfstring{\emph{Statika}: machine configuration and
behavior}{Statika: machine configuration and behavior}}\label{statika-machine-configuration-and-behavior}

\href{https://github.com/ohnosequences/statika}{Statika} is a Scala
library developed by the first and last authors which serves as a way of
defining and composing machine behaviors statically. The main component
are \textbf{bundles}. Each bundle declares a sequence of computations
(its behavior) which will be executed in an \textbf{environment}. A
bundle can \emph{depend} on other bundles, and when being executed by an
environment, its DAG of dependencies is linearized and run in sequence.
In our use, bundles correspond to what an EC2 instance should do and an
environment to an image (AMI: \textbf{A}mazon \textbf{M}achine
\textbf{I}mage) which prepares the basic configuration, downloads the
Scala code and runs it.

\subsubsection{\texorpdfstring{\emph{Datasets}: a mini-language for
data}{Datasets: a mini-language for data}}\label{datasets-a-mini-language-for-data}

\href{https://github.com/ohnosequences/datasets}{Datasets} is a Scala
library developed by the first and last authors with the goal of being a
Scala-embedded mini-language for datasets and their locations.
\textbf{Data} is represented as type-indexed fields: Keys are modeled as
singleton types, and values correspond to what could be called a
denotation of the key: a value of type \texttt{Location} tagged with the
key type. Then a \textbf{Dataset} is essentially a collection of data,
which are guaranteed statically to be different through type-level
predicates, making use of the value ↔ type correspondence which can be
established through singleton types and implicits. A dataset location is
then just a list of locations formed by locations of each data member of
that dataset. All this is based on what could be described as an
embedding in Scala of an extensible record system with concatenation on
disjoint labels, in the spirit of
\citep[\citet{harper1991record}]{harper1990extensible}. For that
\emph{Datasets} uses
\href{https://github.com/ohnosequences/cosas/}{ohnosequences/cosas}.

Data keys can further have a reference to a \textbf{data type}, which,
as the name hints at, can help in providing information about the type
of data we are working with. For example, when declaring Illumina reads
as a data, a data type containing information about the read length,
insert size or end type (single or paired) is used.

A \textbf{location} can be, for example, an S3 object or a local file;
by leaving the location type used to denote particular data free we can
work with different ``physical'' representations, while keeping track of
to which logical data they are a representation of. Thus, a process can
generate locally a \texttt{.fastq} file representing the merged reads,
while another can put it in S3 with the fact that they all correspond to
the ``same'' merged reads is always present, as the data that those
``physical'' representations denote.

\subsubsection{\texorpdfstring{\emph{Loquat}: Parallel data processing
with
AWS}{Loquat: Parallel data processing with AWS}}\label{loquat-parallel-data-processing-with-aws}

\href{https://github.com/ohnosequences/loquat}{Loquat} is a library
developed by the first, second and last authors designed for the
execution of embarrassingly parallel tasks using S3, SQS and EC2.

A \textbf{loquat} executes a process with explicit input and output
datasets (declared using the \emph{Datasets} library described above).
Workers (EC2 instances) read from an SQS queue the S3 locations for both
input and output data; then they download the input to local files, and
pass these file locations to the process to be executed. The output is
then put in the corresponding S3 locations.

A manager instance is used to monitor workers, provide initial data to
be put in the SQS queue and optionally release resources depending on a
set of configurable conditions.

Both worker and manager instances are Statika bundles. In the case of
the worker, it can declare any dependencies needed to perform its task:
other tools, libraries, or data.

All configuration such as the number of workers or the instance types is
declared statically, the specification of a loquat being ultimately a
Scala object. There are deploy and resource management methods, making
it easy to use an existing loquat either as a library or from (for
example) a Scala REPL.

The input and output (and their locations) being defined statically has
several critical advantages. First, composing different loquats is easy
and safe; just use the output types and locations of the first one as
input for the second one. Second, data and their types help in not
mixing different resources when implementing a process, while serving as
a safe and convenient mechanism for writing generic processing tasks.
For example, merging paired-end Illumina reads generically is easy as
the data type includes the relevant information (insert size, read
length, etc) to pass to a tool such as FLASH.

\subsubsection{Type-safe eDSLs for BLAST and
FLASH}\label{type-safe-edsls-for-blast-and-flash}

We developed our own Scala-based type-safe eDSLs (\textbf{e}mbedded
\textbf{D}omain \textbf{S}pecific \textbf{L}anguage) for
\href{https://github.com/ohnosequences/flash}{FLASH} and
\href{https://github.com/ohnosequences/blast}{BLAST} expressions and
their execution.

In the case of BLAST we use a model for expressions where we can
guarantee for each BLAST command expression at compile time

\begin{itemize}
\tightlist
\item
  all required arguments are provided
\item
  only valid options are provided
\item
  correct types for each option value
\item
  valid output record specification
\end{itemize}

Generic type-safe parsers returning an heterogeneous record of BLAST
output fields are also available, together with output data defined
using \emph{Datasets} which have a reference to the exact BLAST command
options which yielded that output. This let us provide generic parsers
for BLAST output which are guaranteed to be correct, for example.

In the same spirit as for BLAST, we implemented a type-safe EDSL for
FLASH expressions and their execution, sporting features equivalent to
those outlined for the BLAST EDSL.

\subsubsection{Bio4j and Graph
Databases}\label{bio4j-and-graph-databases}

\citep[Bio4j][]{pareja2015bio4j} is a data platform integrating data
from different resources such as UniProt or GO in a graph data paradigm.
In the assignment phase we use a subgraph containing the NCBI Taxonomy,
wrapping in Scala its Java API in a tree algebraic data type.

\subsubsection{16S Reference Database
Construction}\label{s-reference-database-construction}

Our 16S Reference Database is a curated subset of sequences from NCBI
nucleotide database \textbf{nt}. The sequences included were selected by
similarity with the bacterial and archaeal reference sequences
downloaded from the \textbf{RDP database} \citep{cole2013ribosomal}. RDP
unaligned sequences were used to capture new 16S sequences from
\textbf{nt} using BLAST similarity search strategies and then,
performing additional curation steps to remove sequences with poor
taxonomic assignments to taxonomic nodes close to the root of the
taxonomy tree. All the nucleotide sequences included in \textbf{nt}
database has a taxonomic assignment provided by the \textbf{Genbank}
sequence submitter. NCBI provides a table (available at
ftp://ftp.ncbi.nlm.nih.gov/pub/taxonomy/) to do the mapping of any
Genbank Identifier (GI) to its Taxonomy Identifier (TaxID). Thus, we are
based on a crowdsourced submitter-maintained taxonomic annotation system
for reference sequences. It supposes a sustainable system able to face
the expected number of reference sequences that will populate the public
global nucleotide databases in the near future. Another advantageous
point is that we are based on NCBI taxonomy, the \emph{de facto}
standard taxonomic classification for biomolecular data
\citep{cochrane20102010}. NCBI taxonomy is, undoubtedly, the most used
taxonomy all over the world and the most similar to the official
taxonomies of each specific field. This is a crucial point because all
the type-culture and tissue databanks follow this official taxonomical
classification and, in addition, all the knowledge accumulated during
last decades is referred to this taxonomy. In addition NCBI provides a
direct connection between taxonomical formal names and the physical
specimens that serve as exemplars for the species
\citep{federhen2014type}.

Certainly, if metagenomics results are easily integrated with the
theoretical and experimental knowledge of each specific area, the impact
of metagenomics will be higher that if metagenomics progresses as a
disconnected research branch. Considering that metagenomics data
interoperability, which is especially critical in clinical environments,
requires a stable taxonomy to be used as reference, we decided to rely
on the most widely used taxonomy: the NCBI taxonomy. In addition, the
biggest global sequence database GenBank follows this taxonomy to
register the origin of all their submitted sequences. Our 16S database
building strategy allows the substitution of the 16S database by any
other subset of \textbf{nt}, even by the complete \textbf{nt} database
if it would be needed, for example, for analyzing shotgun metagenomics
data. This possibility of changing the reference database provides
flexibility to the system enabling it for easy updating and
project-driven personalization.

\subsection{Workflow Description}\label{workflow-description}

The MG7 analysis workflow is summarized in Figure X. The input files for
MG7 are the FASTQ files resulting from a paired-end NGS sequencing
experiment.

\subsubsection{Joining reads of each pair using
FLASH}\label{joining-reads-of-each-pair-using-flash}

In the first step the paired-end reads, designed with an insert size
that yields pairs of reads with an overlapping region between them, are
assembled using FLASH \citep{magovc2011flash}. FLASH is designed to
merge pairs of reads when the original DNA fragments are shorter than
twice the length of reads. Thus, the sequence obtained after joining the
2 reads of each pair is larger and has better quality since the sequence
at the ends of the reads is refined merging both ends in the assembly.
To have a larger and improved sequence is crucial to do more precise the
inference of the bacterial origin based on similarity with reference
sequences.

\subsubsection{Parallelized BLASTN of each read against the
16S-DB7}\label{parallelized-blastn-of-each-read-against-the-16s-db7}

The second step is to search for similar 16S sequences in our 16S-DB7
database. The taxonomic assignment for each read is based on BLASTN of
each read against the 16S database. Assignment based on direct
similarity of each read one by one compared against a sufficiently wide
database is a very exhaustive method for assignment
\citep{segata2013computational} . Some methods of assignment compare the
sequences only against the available complete bacterial genomes or avoid
computational cost clustering or binning the sequences first, and then
doing the assignments only for the representative sequence of each
cluster. MG7 carries out an exhaustive comparison of all the reads under
analysis and it does not applies any binning strategy. Every read is
specifically compared with all the sequences of the 16S database. We
select the best BLAST hits (10 hits by default) obtained for each read
to do the taxonomic assignment.

\subsubsection{Taxonomic Assignment
Algorithms}\label{taxonomic-assignment-algorithms}

All the reads are assigned under two different algorithms of assignment:
i. Lowest Common Ancestor based taxonomic assignment (LCA) and ii. Best
BLAST Hit based taxonomic assignment (BBH). Figure X displays
schematically the LCA algorithm applied sensu stricto (left panel) and
the called `in line' exception (right panel) designed in order to gain
specificity in the assignments in the cases in which the topology of the
taxonomical nodes corresponding to the BLAST hits support sufficiently
the assignment to the most specific taxon.

\paragraph{Lowest Common Ancestor based Taxonomic
Assignment}\label{lowest-common-ancestor-based-taxonomic-assignment}

For each read, first, we select the BEST BLAST HITs (by default 10 Hits)
over a threshold of similarity (by default \(evalue \leq e^{-15}\))
filtering those hits that are not sufficiently good comparing them with
the best one. We select the best HSP (High Similarity Pair) per
reference sequence and then choose the best HSP (that with lowest e
value) between all the selected ones. The bitscore of this best HSP
(called S) is used as reference to filter the rest of HSPs. All the HSPs
with bitscore below p x S are filtered. p is a coefficient fixed by the
user to define the bitscore required, e.g.~if p=0.9 and S=700 the
required bitscore threshold would be 630. Once we have the definitive
HSPs selected, we obtain their corresponding taxonomic nodes using the
taxonomic assignments that NCBI provides for all the nt database
sequences. Now we have to analyze the topological distribution of these
nodes in the taxonomy tree: i. If all the nodes forms a line in the
taxonomy tree (are located in a not branched lineage to the tree root)
we should choose the most specific taxID as the final assignment for
that read. We call to this kind of assignment the `in line' exception
(see Figure X right panel). ii. If not, we should search for the
\emph{sensu stricto} Lowest Common Ancestor (LCA) of all the selected
taxonomic nodes (See Figure X left panel). In this approach we decided
to use the bitscore for evaluating the similarity because it is a value
that increases when similarity is higher and depends a lot on the length
of the HSP. Some reads could not find sequences with enough similarity
in the database and then they would be classified as reads with no hits.
Advanced metagenomics analysis approaches \citep{huson2013microbial}
have adopted LCA assignment algorithms because it provides fine and
trusted taxonomical assignment.

\paragraph{Best BLAST hit taxonomic
assignment}\label{best-blast-hit-taxonomic-assignment}

We decided to maintain the simpler method of Best BLAST Hit (BBH) for
taxonomic assignment because, in some cases, it can provide information
about the sequences that adds information to that obtained using LCA
algorithm. Using LCA algorithm, when some reference sequences with BLAST
alignments over the required thresholds map to a not sufficiently
specific taxID, the read can be assigned to an unspecific taxon near to
the root of the taxonomy tree. If the BBH reference sequence maps to
more specific taxa, this method, in that case, gives us useful
information.

\subsubsection{Output for LCA and BBH
assignments}\label{output-for-lca-and-bbh-assignments}

MG7 provides independent results for the 2 different approaches, LCA and
BBH. The output files include, for each taxonomy node (with some read
assigned), abundance values for direct assignment and cumulative
assignment. The abundances are provided in counts (absolute values) and
in percentage normalized to the number of reads of each sample. Direct
assignments are calculated counting reads specifically assigned to a
taxonomic node, not including the reads assigned to the descendant nodes
in the taxonomy tree. Cumulative assignments are calculated including
the direct assignments and also the assignments of the descendant nodes.
For each sample MG7 provides 8 kinds of abundance values: LCA direct
counts, LCA cumu. counts, LCA direct \%, LCA cumu. \%, BBH direct
counts, BBH cumu. counts, BBH direct \%, BBH cumu. \%.

\subsection{Data analysis as a software
project}\label{data-analysis-as-a-software-project}

\textbf{IDEAS}

\begin{itemize}
\tightlist
\item
  the user needs to write code, a repo would be nice, AWS account
\item
  there's some sort of conf required for the AWS account (add users,
  roles, permissions, whatever)
\item
  Why this is a good thing (or should this just be in Discussion)
\item
  something else?
\end{itemize}

\subsection{Availability}\label{availability}

MG7 is open source, available at https://github.com/ohnosequences/mg7
under an \href{http://www.gnu.org/licenses/agpl-3.0.en.html}{AGPLv3}
license.

\section{Discussion}\label{discussion}

\subsection{What MG7 brings}\label{what-mg7-brings}

We could summarize the most innovative ideas and developments in MG7:

\begin{enumerate}
\def\labelenumi{\arabic{enumi}.}
\tightlist
\item
  Treat data analysis as a software project. This makes for radical
  improvements in \emph{reproducibility}, \emph{reuse},
  \emph{versioning}, \emph{safety}, \emph{automation} and
  \emph{expressiveness}
\item
  input and output data, their locations and type are expressible and
  checked at compile-time using \emph{Datasets} ­3. management of
  dependencies and machine configurations using \emph{Statika}
\item
  automation of AWS cloud resources and processes, including
  distribution and parallelization through the use of \emph{Loquat}
\item
  taxonomic data and related operations are treated natively as what
  they are: graphs, through the use of \emph{Bio4j}
\item
  MG7 provides a sustainable model for taxonomic assignment, appropriate
  to face the challenging amount of data that high throughput sequencing
  technologies generate
\end{enumerate}

We will expand on each item in the following sections.

\subsection{A new approach to data
analysis}\label{a-new-approach-to-data-analysis}

MG7 proposes to define and work with a particular data analysis task as
a software project, using Scala. The idea is that \emph{everything}:
data description, their location, configuration parameters, the
infrastructure used, \ldots{} should be expressed as Scala code, and
treated in the same way as any (well-managed) software project. This
includes, among other things, using version control systems
(\texttt{git} in our case), writing tests, making stable releases
following \href{http://semver.org/}{semantic versioning} or publishing
artifacts to a repository.

What we see as key advantages of this approach (when coupled with
compile-time specification and checking), are

\begin{itemize}
\tightlist
\item
  \textbf{Reproducibility} the same analysis can be run again with
  exactly the same configuration in a trivial way.
\item
  \textbf{Versioning} as in any software project, there can be different
  versions, stable releases, etc.
\item
  \textbf{Reuse} we can build standard configurations on top of this and
  reuse them for subsequent data analysis. A particular data analysis
  \emph{task} can be used as a \emph{library} in further analysis.
\item
  \textbf{Decoupling} We can start working on the analysis
  specification, without any need for available data in a much easier
  way.
\item
  \textbf{Documentation} We can take advantage of all the effort put
  into software documentation tools and practices, such as in our case
  Scaladoc or literate programming. As documentation, analysis processes
  and data specification live together in the files, it is much easier
  to keep coherence between them.
\item
  \textbf{Expresiveness and safety} For example in our case we can
  choose only from valid Illumina read types, and then build a default
  FLASH command based on that. The output locations, being declared
  statically, are also available for use in further analysis.
\end{itemize}

\subsection{Inputs, outputs, data: compile-time, expressive,
composable}\label{inputs-outputs-data-compile-time-expressive-composable}

\subsection{Tools, data, dependencies and machine
configurations}\label{tools-data-dependencies-and-machine-configurations}

\subsection{Parallel cloud execution ??}\label{parallel-cloud-execution}

\subsection{Taxonomy and Bio4j}\label{taxonomy-and-bio4j}

The hierarchic structure of the taxonomy of the living organisms is a
tree, and, hence, is also a graph in which each node, with the exception
of the root node, has a unique parent node. It led us to model the
taxonomy tree as a graph using the graph database paradigm. Previously
we developed Bio4j \textbf{{[}Pareja-Tobes-2015{]}}, a platform for the
integration of semantically rich biological data using typed graph
models. It integrates most publicly available data linked with sequences
into a set of interdependent graphs to be used for bioinformatics
analysis and especially for biological data.

\subsection{Future-proof}\label{future-proof}

\subsection{MG7 Future developments}\label{mg7-future-developments}

\subsubsection{Shotgun metagenomics}\label{shotgun-metagenomics}

It is certainly possible to adapt MG7 to work with shotgun metagenomics
data. Simply changing the reference database to include whole genome
sequence data could yield interesting results. This could also be
refined by restricting reference sequences according to all sort of
criteria, like biological function or taxonomy. Bio4j would be an
invaluable tool here, thanks to its ability to express express complex
predicates on sequences using all the information linked with them (GO
annotations, UniProt data, NCBI taxonomy, etc).

\subsubsection{Comparison of groups of
samples}\label{comparison-of-groups-of-samples}

\subsubsection{Interactive visualizations using the output files of MG7
(Biographika
project)}\label{interactive-visualizations-using-the-output-files-of-mg7-biographika-project}

\section{Materials and Methods}\label{materials-and-methods}

\subsection{Amazon Web Services}\label{amazon-web-services}

We use EC2, S3 and SQS through a Scala wrapper of the official
\href{https://aws.amazon.com/sdk-for-java/}{AWS Java SDK},
\href{https://github.com/ohnosequences/aws-scala-tools/???}{ohnosequences/aws-scala-tools
0.13.2}. This uses version \texttt{1.10.9} of the AWS Java SDK.

\subsection{Scala}\label{scala}

MG7 itself and all the libraries used are written in Scala
\texttt{2.11}.

\subsection{Statika}\label{statika}

MG7 uses
\href{https://github.com/statika/statika/releases/tag/v0.2.0}{ohnosequences/statika
2.0.0} for specifying the configuration and behavior of EC2 instances.

\subsection{Datasets}\label{datasets}

MG7 uses
\href{https://github.com/ohnosequences/datasets/releases/tag/v0.2.0}{ohnosequences/datasets
0.2.0} for specifying input and output data, their type and their
location.

\subsection{Loquat}\label{loquat}

MG7 uses
\href{https://github.com/ohnosequences/loquat/releases/tag/v2.0.0}{ohnosequences/loquat
2.0.0} for the specification of data processing tasks and their
execution using AWS resources.

\subsection{BLAST eDSL}\label{blast-edsl}

MG7 uses
\href{https://github.com/ohnosequences/blast/releases/tag/v0.2.0}{ohnosequences/blast
0.2.0}. The BLAST version used is \texttt{2.2.31+}

\subsection{FLASH eDSL}\label{flash-edsl}

MG7 uses
\href{https://github.com/ohnosequences/flash/releases/tag/v0.1.0}{ohnosequences/flash
0.1.0}. The FLASH version used is \texttt{?.?.?}

\subsection{Bio4j}\label{bio4j}

MG7 uses
\href{https://github.com/bio4j/bio4j/releases/tag/v0.12.0-RC3}{bio4j/bio4j
0.12.0-RC3} and
\href{https://github.com/bio4j/bio4j-titan/releases/tag/v0.4.0-RC2}{bio4j/bio4j-titan
0.4.0-RC2} as an API for the NCBI taxonomy.

\section{Acknowledgements}\label{acknowledgements}

\emph{Funding:} The two first authors are funded by INTERCROSSING (Grant
289974).

\bibliographystyle{frontiersinSCNS_ENG_HUMS} % for Science, Engineering and Humanities and Social Sciences articles, for Humanities and Social Sciences articles please include page numbers in the in-text citations
%\bibliographystyle{frontiersinHLTH&FPHY} % for Health and Physics articles
\bibliography{refs}

\end{document}
