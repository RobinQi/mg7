%%%%%%%%%%%%%%%%%%%%%%%%%%%%%%%%%%%%%%%%%%%%%%%%%%%%%%%%%%%%%%%%%%%%%%%%%%%%%%%%%%%%%%%%%%%%%%%%%%%%%%%%%%%%%%%%%%%%%%%%%%%%%%%%%%%%%%%%%%%%%%%%%%%%%%%%%%%
% This is just an example/guide for you to refer to when submitting manuscripts to Frontiers, it is not mandatory to use Frontiers .cls files nor frontiers.tex  %
% This will only generate the Manuscript, the final article will be typeset by Frontiers after acceptance.                                                 %
%                                                                                                                                                         %
% When submitting your files, remember to upload this *tex file, the pdf generated with it, the *bib file (if bibliography is not within the *tex) and all the figures.
%%%%%%%%%%%%%%%%%%%%%%%%%%%%%%%%%%%%%%%%%%%%%%%%%%%%%%%%%%%%%%%%%%%%%%%%%%%%%%%%%%%%%%%%%%%%%%%%%%%%%%%%%%%%%%%%%%%%%%%%%%%%%%%%%%%%%%%%%%%%%%%%%%%%%%%%%%%

%%% Version 3.1 Generated 2015/22/05 %%%
%%% You will need to have the following packages installed: datetime, fmtcount, etoolbox, fcprefix, which are normally inlcuded in WinEdt. %%%
%%% In http://www.ctan.org/ you can find the packages and how to install them, if necessary. %%%

\documentclass{frontiersSCNS} % for Science, Engineering and Humanities and Social Sciences articles
%\documentclass{frontiersHLTH} % for Health articles
%\documentclass{frontiersFPHY} % for Physics and Applied Mathematics and Statistics articles

%\setcitestyle{square}
\usepackage{url,hyperref,lineno,microtype}
\usepackage[onehalfspacing]{setspace}
\linenumbers

% fix for pandoc 1.14
\providecommand{\tightlist}{%
\setlength{\itemsep}{0pt}\setlength{\parskip}{0pt}}

% Leave a blank line between paragraphs instead of using \\


\def\keyFont{\fontsize{8}{11}\helveticabold }
\def\firstAuthorLast{Alexey Alekhin {et~al.}} %use et al only if is more than 1 author
\def\Authors{
 Alexey Alekhin\,$^{1}$  Evdokim Kovach\,$^{1}$  Marina Manrique\,$^{1}$  Pablo Pareja\,$^{1}$  Eduardo Pareja\,$^{1}$  Raquel Tobes\,$^{1}$  and Eduardo Pareja-Tobes $^{1,*}$
}% Affiliations should be keyed to the author's name with superscript numbers and be listed as follows: Laboratory, Institute, Department, Organization, City, State abbreviation (USA, Canada, Australia), and Country (without detailed address information such as city zip codes or street names).
% If one of the authors has a change of address, list the new address below the correspondence details using a superscript symbol and use the same symbol to indicate the author in the author list.
\def\Address{$^{1}$Oh no sequences! Research Group, Era7 Bioinformatics, Granada, Spain}
% The Corresponding Author should be marked with an asterisk
% Provide the exact contact address (this time including street name and city zip code) and email of the corresponding author
\def\corrAuthor{Corresponding Author}
\def\corrAddress{Oh no sequences! Research Group, Era7 Bioinformatics, Plaza Campo Verde 3, Granada, 18001, Spain}
\def\corrEmail{eparejatobes@ohnosequences.com}


\begin{document}
\onecolumn
\firstpage{1}

\title[MG7: Configurable and scalable 16S metagenomics data analysis]{MG7: Configurable and scalable 16S metagenomics data analysis -- new
methods optimized for massive cloud computing}

\author[\firstAuthorLast ]{\Authors} %This field will be automatically populated
\address{} %This field will be automatically populated
\correspondance{} %This field will be automatically populated

\extraAuth{}% If there are more than 1 corresponding author, comment this line and uncomment the next one.
%\extraAuth{corresponding Author2 \\ Laboratory X2, Institute X2, Department X2, Organization X2, Street X2, City X2 , State XX2 (only USA, Canada and Australia), Zip Code2, X2 Country X2, email2@uni2.edu}

\maketitle

\begin{abstract}

No abstract yet. Will be here.

\tiny
 \keyFont{ \section{Keywords:} Metagenomics, 16S, Bacterial diversity profile, Bio4j, Graph databases,
Cloud computing, NGS, Genomic big data } %All article types: you may provide up to 8 keywords; at least 5 are mandatory.
\end{abstract}

\section{Introduction}\label{introduction}

Metagenomics data analysis is growing at exponential rate during the
last years. The increasing throughput of massively parallel sequencing
technologies, the derived decreasing cost, and the high impact of
metagenomics studies, especially in human health (diagnostics,
treatments, drug response, prevention), are crucial reasons responsible
for this growth of Metagenomics. There is a growing interest in
sequencing all kind of microbiomes (gut, mouth, skin, urinary tract,
airway, milk, bladder), in different conditions of health and disease,
or after different treatments. Metagenomics is also impacting
environmental sciences, crop sciences, agrifood sector and biotechnology
in general. This new possibilities for exploring the diversity of
micro-organisms in the most diverse environments is opening many new
research areas but, due to this wide interest, it is expected that the
amount of data will be overwhelming in the short time
\citep{stephens2015big}.

Genome researchers have raised the alarm over big data in the past
\href{http://www.nature.com/news/genome-researchers-raise-alarm-over-big-data-1.17912}{nature
news add ref} but even a more serious challenge might be faced with the
metagenomics boom/ upswing. If we compare metagenomics data with other
genomics data used in clinical genotyping we find a differential
feature: the key role of time. Thus, for example, in some longitudinal
studies, serial sampling of the same patient along several weeks (or
years) is being used for the follow up of some intestinal pathologies,
for studying the evolution of gut microbiome after antibiotic treatment,
or for colon cancer early detection \citep{zeller2014potential}. This
need of sampling across time adds more complexity to metagenomics data
storage and demands adapted algorithms to detect state variations across
time as well as idiosyncratic commonalities of the microbiome of each
individual \citep{franzosa2015identifying}. In addition to the
intra-individual sampling-time dependence, metagenomic clinical test
results vary depending on the specific region of extraction of the
clinical specimen. This local variability adds complexity to the
analysis since different localizations (different tissues, different
anatomical regions, healthy or tumour tissues) are required to have a
sufficiently complete landscape of the human microbiome. Moreover,
reanalysis of old samples using new tools and better reference databases
might be also demanded from time to time.

During the last years other sciences as astronomy or particle physics
are facing the big data challenge but, at least, these science have
standards for data processing \citep{stephens2015big}. Global standards
for converting raw sequence data into processed data are not yet well
defined in metagenomics and there are shortcomings derived from the fact
that many bioinformatics methodologies currently used for metagenomics
data analysis were designed for a scenario very different that the
current one. These are some of the aspects that have suffered crucial
changes and advances with a direct impact in metagenomics data analysis.
i. The first aspect is related to the sequences to be analyzed: the
reads are larger, the sequencing depth and the number of samples of each
project are considerably bigger. The first metagenomics studies were
very local projects, while nowadays the most fruitful studies are done
at a global level (international, continental, national). This kind of
global studies has yielded the discovery of clinical biomarkers for
diseases of the importance of cancer, obesity or inflammatory bowel
diseases and has allowed exploring the biodiversity in many earth
environments ii. The second aspect derives from the impressive genomics
explosion, its effect being felt in this case in the reference
sequences. The immense amount of sequences available in public
repositories demands new approaches in curation, update and storage for
metagenomics reference databases: current models will or already have
problems to face the future avalanche of metagenomic sequences. iii. The
third aspect to consider for metagenomics data analysis is related to
the appearance of new models for massive computation and storage and to
the new programming methodologies (Scala, \ldots{}) and new cloud models
and resources. The immense new possibilities that these advances offer
must have a direct impact in the metagenomics data analysis. iv. And
finally the new social manner to do science, and especially genomic
science is the fourth aspect to consider. Metagenomics evolves in a
social and global scenario following a science democratization trend in
which many small research groups from distant countries share a common
big metagenomics project. This global cooperation demands systems
allowing following exactly the same pipelines using equivalent cloud
resources to modularly execute the analysis in an asynchronous way of
working between different groups. This definitively new scenario demands
new methods and tools to handle the current and future volume of
metagenomic data with the sufficient speed of analysis. Considering all
these aspects we have designed a new open source methodology for
analyzing metagenomics data that exploits the new possibilities that
cloud computing offers to get a system robust, programmatically
configurable, modular, distributed, flexible, scalable and traceable in
which the biological databases of reference sequences can be easily
updated and/or frequently substituted by new ones or by databases
specifically designed for focused projects.

\section{Materials and Methods}\label{materials-and-methods}

\subsection{Amazon Web Services}\label{amazon-web-services}

\subsection{Scala}\label{scala}

\href{http://www.scala-lang.org/}{Scala} is a hybrid object-functional
programming language which runs on Java Virtual Machine. It has support
for type-level programming, type-dependent types (through type members)
and singleton types, which permits a restricted form of dependent types
where types can depend essentially on values determined at compile time
(through their corresponding singleton types). Conversely, through
implicits one can retrieve the value corresponding to a singleton type.

The other key feature for us is Java interoperability, which let us
build on the vast number of existing Java libraries; we take advantage
of this when using Bio4j as an API for the NCBI taxonomy.

MG7 itself and all the libraries used are written in Scala
\texttt{2.11}.

\subsection{Statika}\label{statika}

\href{https://github.com/ohnosequences/statika}{Statika} is a Scala
library developed by \textbf{so and so} which serves as a way of
defining and composing machine behaviors statically. The main component
are \textbf{bundles}. Each bundle declares a sequence of computations
(its behavior) which will be executed in an \textbf{environment}. A
bundle can \emph{depend} on other bundles, and when being executed by an
environment, its DAG of dependencies is linearized and run in sequence.
In our use, bundles correspond to what an EC2 instance should do and an
environment to an image (AMI: \textbf{A}mazon \textbf{M}achine
\textbf{I}mage) which prepares the basic configuration, downloads the
Scala code and runs it.

\subsection{Datasets}\label{datasets}

\href{https://github.com/ohnosequences/datasets}{Datasets} is a Scala
library developed by \textbf{so and so} to declare datasets and their
locations. \textbf{Data} is represented as type-indexed fields: Keys are
modeled as singleton types, and values correspond to what could be
called a denotation of the key: a value of type \texttt{Location} tagged
with the key type. Then a \textbf{Dataset} is essentially a collection
of data, which are guaranteed statically to be different through
type-level predicates, making use of the value ↔ type correspondence
which can be established through singleton types and implicits. A
dataset location is then just a list of locations formed by locations of
each data member of that dataset.

Data keys can further have a reference to a \textbf{data type}, which,
as the name hints at, can help in providing information about the type
of data we are working with. For example, when declaring Illumina reads
as a data, a data type containing information about the read length,
insert size or end type (single or paired) is used.

A \textbf{location} can be, for example, an S3 object or a local file;
by leaving the location type used to denote particular data free we can
work with different ``physical'' representations, while keeping track of
to which logical data they are a representation of. Thus, a process can
generate locally a \texttt{.fastq} file representing the merged reads,
while another can put it in S3 with the fact that they all correspond to
the ``same'' merged reads is always present, as the data that those
``physical'' representations denote.

\subsection{Loquat}\label{loquat}

\href{https://github.com/ohnosequences/loquat}{Loquat} is a library
developed by \textbf{so and so} designed for the execution of
embarrassingly parallel tasks using S3, SQS and EC2.

A \textbf{loquat} executes a process with explicit input and output
datasets (declared using the \emph{Datasets} library described above).
Workers (EC2 instances) read from an SQS queue the S3 locations for both
input and output data; then they download the input to local files, and
pass these file locations to the process to be executed. The output is
then put in the corresponding S3 locations.

A manager instance is used to monitor workers, provide initial data to
be put in the SQS queue and optionally release resources depending on a
set of configurable conditions.

Both worker and manager instances are Statika bundles. In the case of
the worker, it can declare any dependencies needed to perform its task:
other tools, libraries, or data.

All configuration such as the number of workers or the instance types is
declared statically, the specification of a loquat being ultimately a
Scala object. There are deploy and resource management methods, making
it easy to use an existing loquat either as a library or from (for
example) a Scala REPL.

The input and output (and their locations) being defined statically has
several critical advantages. First, composing different loquats is easy
and safe; just use the output types and locations of the first one as
input for the second one. Second, data and their types help in not
mixing different resources when implementing a process, while serving as
a safe and convenient mechanism for writing generic processing tasks.
For example, merging paired-end Illumina reads generically is easy as
the data type includes the relevant information (insert size, read
length, etc) to pass to a tool such as FLASH.

\subsection{Type-safe DSLs for BLAST and
FLASH}\label{type-safe-dsls-for-blast-and-flash}

We developed our own type-safe DSLs (Domain Specific Language) for
\href{https://github.com/ohnosequences/flash}{FLASH} and
\href{https://github.com/ohnosequences/blast}{BLAST} expressions and
their execution.

\subsubsection{BLAST DSL}\label{blast-dsl}

In the case of BLAST we use a model for expressions where we can
guarantee for each BLAST command expression at compile time

\begin{itemize}
\tightlist
\item
  all required arguments are provided
\item
  only valid options are provided
\item
  correct types for each option value
\item
  valid output record specification
\end{itemize}

Generic type-safe parsers returning an heterogeneous record of BLAST
output fields are also available, together with output data defined
using \emph{Datasets} which have a reference to the exact BLAST command
options which yielded that output. This let us provide generic parsers
for BLAST output which are guaranteed to be correct, for example.

\subsubsection{FLASH DSL}\label{flash-dsl}

In the same spirit as for BLAST,

\subsection{Bio4j}\label{bio4j}

\href{https://github.com/bio4j/bio4j}{Bio4j} is a data platform
integrating data from different resources such as UniProt or GO in a
graph data paradigm. We use the module containing the NCBI Taxonomy, and
the use their Java API from Scala in the assignment phase.

\section{Results}\label{results}

\subsection{Overview}\label{overview}

To tackle the challenges posed by metagenomics big data analysis
outlined in the Introduction,

­- AWS resources in Scala (??) - A new approach to data analysis
specification, management and specification based on working with it in
exactly the same way as for a software project, together with the
extensive use of compile-time structures and checks. - Parallelization
and distributed analysis based on AWS, with on-demand infrastructure as
the basic paradigm ­- fully automated processes, data and cloud
resources management. - Static reproducible specification of
dependencies and behavior of the different components using
\emph{Statika} and \emph{Datasets} - Definition of complex pipelines
using \emph{Loquat} a composable system for scaling/parallelizing
stateless computations especially designed for Amazon Web Services (AWS)
- Modeling of the taxonomy tree using the new paradigm of graph
databases (Bio4j). It facilitates the taxonomic assignment tasks and the
calculation of the taxa abundance values considering the hierarchic
structure of taxonomy tree (cumulative values). - per-read assignment
(??)

\subsection{16S Reference Database
Construction}\label{s-reference-database-construction}

Our 16S Reference Database is a curated subset of sequences from NCBI
nucleotide database \textbf{nt}. The sequences included were selected by
similarity with the bacterial and archaeal reference sequences
downloaded from the \textbf{RDP database} \citep{cole2013ribosomal}. RDP
unaligned sequences were used to capture new 16S sequences from
\textbf{nt} using BLAST similarity search strategies and then,
performing additional curation steps to remove sequences with poor
taxonomic assignments to taxonomic nodes close to the root of the
taxonomy tree. All the nucleotide sequences included in \textbf{nt}
database has a taxonomic assignment provided by the \textbf{Genbank}
sequence submitter. NCBI provides a table (available at
ftp://ftp.ncbi.nlm.nih.gov/pub/taxonomy/) to do the mapping of any
Genbank Identifier (GI) to its Taxonomy Identifier (TaxID). Thus, we are
based on a crowdsourced submitter-maintained taxonomic annotation system
for reference sequences. It supposes a sustainable system able to face
the expected number of reference sequences that will populate the public
global nucleotide databases in the near future. Another advantageous
point is that we are based on NCBI taxonomy, the \emph{de facto}
standard taxonomic classification for biomolecular data
\citep{cochrane20102010}. NCBI taxonomy is, undoubtedly, the most used
taxonomy all over the world and the most similar to the official
taxonomies of each specific field. This is a crucial point because all
the type-culture and tissue databanks follow this official taxonomical
classification and, in addition, all the knowledge accumulated during
last decades is referred to this taxonomy. In addition NCBI provides a
direct connection between taxonomical formal names and the physical
specimens that serve as exemplars for the species
\citep{federhen2014type}.

Certainly, if metagenomics results are easily integrated with the
theoretical and experimental knowledge of each specific area, the impact
of metagenomics will be higher that if metagenomics progresses as a
disconnected research branch. Considering that metagenomics data
interoperability, which is especially critical in clinical environments,
requires a stable taxonomy to be used as reference, we decided to rely
on the most widely used taxonomy: the NCBI taxonomy. In addition, the
biggest global sequence database GenBank follows this taxonomy to
register the origin of all their submitted sequences. Our 16S database
building strategy allows the substitution of the 16S database by any
other subset of \textbf{nt}, even by the complete \textbf{nt} database
if it would be needed, for example, for analyzing shotgun metagenomics
data. This possibility of changing the reference database provides
flexibility to the system enabling it for easy updating and
project-driven personalization.

\subsection{Bio4j and Graph Databases}\label{bio4j-and-graph-databases}

\subsection{MG7 Pipeline Description}\label{mg7-pipeline-description}

\subsection{Taxonomic Assignment
Algorithms}\label{taxonomic-assignment-algorithms}

\subsubsection{Lowest Common Ancestor based Taxonomic
Assignment}\label{lowest-common-ancestor-based-taxonomic-assignment}

For each read:

­1. Select only one BLASTN alignment (HSP) per reference sequence (the
HSP with lowest e value) 2. Filter all the HSPs with bitscore below a
defined BLASTN bitscore threshold s\_0 3. Find the best bitscore value S
in the set of BLASTN HSPs corresponding to hits of that read 4. Filter
all the alignments with bitscore below p * S (where p is a fixed by the
user coefficient to define the bitscore required, e.g.~if p=0.9 and
S=700 the required bitscore threshold would be 630) 5. Select all the
taxonomic nodes to which map the reference sequences involved in the
selected HSPs: - If all the selected taxonomic nodes forms a line in the
taxonomy tree (are located in a not branched lineage to the tree root)
we should choose the most specific taxID as the final assignment for
that read - If not, we should search for the (sensu stricto) Lowest
Common Ancestor (LCA) of all the selected taxonomic nodes (See Figure X)

In this approach the value used for evaluating the similarity is the
bitscore that is a value that increases when similarity is higher and
depends a lot on the length of the HSP

\subsubsection{Best BLAST hit taxonomic
assignment}\label{best-blast-hit-taxonomic-assignment}

We have maintained the simpler method of Best BLAST Hit (BBH) taxonomic
assignment because, in some cases, it can provide information about the
sequences that can be more useful than the obtained using LCA algorithm.
Using LCA algorithm when some reference sequences with BLAST alignments
over the required thresholds map to a not sufficiently specific taxID,
the read can be assigned to an unspecific taxon near to the root. If the
BBH reference sequence maps to a more specific taxa this method, in that
case, gives us useful information.

\subsection{Using MG7 with some example
data-sets}\label{using-mg7-with-some-example-data-sets}

We selected the datasets described in {[}Kennedy-2014{]} (??)

\subsection{MG7 availability}\label{mg7-availability}

MG7 is open source, available at https://github.com/ohnosequences/mg7
under an \href{http://www.gnu.org/licenses/agpl-3.0.en.html}{AGPLv3}
license.

\section{Discussion}\label{discussion}

\subsection{What MG7 brings}\label{what-mg7-brings}

We could summarize the most innovative ideas and developments in MG7:

\begin{enumerate}
\def\labelenumi{\arabic{enumi}.}
\tightlist
\item
  Treat data analysis as a software project. This makes for radical
  improvements in \emph{reproducibility}, \emph{reuse},
  \emph{versioning}, \emph{safety}, \emph{automation} and
  \emph{expressiveness}
\item
  input and output data, their locations and type are expressible and
  checked at compile-time using our Scala library \emph{datasets} ­3.
  management of dependencies and machine configurations using our Scala
  library \emph{Statika}
\item
  automation of AWS cloud resources and processes, including
  distribution and parallelization through the use of \emph{Loquat}
\item
  taxonomic data and related operations are treated natively as what
  they are: graphs, through the use of \emph{Bio4j}
\item
  MG7 provides a sustainable model for taxonomic assignment, appropriate
  to face the challenging amount of data that high throughput sequencing
  technologies generate
\end{enumerate}

\subsection{A new approach to data
analysis}\label{a-new-approach-to-data-analysis}

General approach. An analysis is defined as a software project. It can
evolve in the same way. We can run the analysis in a test phase, review
configuration and changes, etc. Key advantages of this approach are

\begin{itemize}
\tightlist
\item
  \textbf{Reproducibility} the same analysis can be run again with
  exactly the same configuration in a trivial way.
\item
  \textbf{Versioning} The analysis is a software project so it goes
  through the same stages, there can be different versions, stable
  releases, etc.
\item
  \textbf{Reuse} we can build standard configurations on top of this and
  reuse them for subsequent data analysis.
\item
  \textbf{Decoupling} We can start working on the analysis
  specification, without any need for data in a much easier way.
\item
  \textbf{Expresiveness and safety} choose only from valid Illumina read
  types, build default FLASH command based on that, \ldots{}
\end{itemize}

\subsection{Inputs, outputs, data: compile-time, expressive,
composable}\label{inputs-outputs-data-compile-time-expressive-composable}

\subsection{Tools, data, dependencies and machine
configurations}\label{tools-data-dependencies-and-machine-configurations}

\subsection{Parallel cloud execution ??}\label{parallel-cloud-execution}

\subsection{Taxonomy and Bio4j}\label{taxonomy-and-bio4j}

\subsection{Future-proof}\label{future-proof}

\subsection{MG7 Future developments}\label{mg7-future-developments}

Other possible uses of the general schema: statika, loquat, \ldots{}

\subsubsection{Comparison of groups of
samples}\label{comparison-of-groups-of-samples}

\subsubsection{Interactive visualizations using the output files of MG7
(Biographika
project)}\label{interactive-visualizations-using-the-output-files-of-mg7-biographika-project}

\section{Acknowledgements}\label{acknowledgements}

INTERCROSSING (Grant 289974)

\bibliographystyle{frontiersinSCNS_ENG_HUMS} % for Science, Engineering and Humanities and Social Sciences articles, for Humanities and Social Sciences articles please include page numbers in the in-text citations
%\bibliographystyle{frontiersinHLTH&FPHY} % for Health and Physics articles
\bibliography{refs}

\end{document}
